% =========================================
% Linker 學習歷程檔案 · XeLaTeX(Overleaf 直接可編)
% =========================================
\documentclass[12pt,a4paper]{article}

% ---------- 版面與文字 ----------
\usepackage{geometry}
\geometry{margin=25mm}
\usepackage{fontspec}
\usepackage{xeCJK}
\usepackage{microtype}
\usepackage{graphicx}
\graphicspath{{figs/}} % 將圖片放在 figs/ 底下(可自行調整)
\xeCJKsetup{CJKglue=\hspace{0.08em}} % 中英間距微調

% 西文字體 + 等寬字體
\setmainfont{TeX Gyre Pagella}
\setsansfont{TeX Gyre Heros}
\setmonofont{Inconsolata}

% 中文字體(Overleaf 常見 Noto;若無則退到 Fandol)
\IfFontExistsTF{Noto Serif CJK TC}{
  \setCJKmainfont{Noto Serif CJK TC}
  \setCJKsansfont{Noto Sans CJK TC}
}{
  \setCJKmainfont{FandolSong}
  \setCJKsansfont{FandolHei}
}

% ---------- 顏色 / 風格 ----------
\usepackage{xcolor}
\definecolor{Brand}{HTML}{0F766E}
\definecolor{Accent}{HTML}{2563EB}
\definecolor{SoftGray}{HTML}{64748B}
\definecolor{LightBg}{HTML}{F1F5F9}

% 品牌宏:統一名字大小寫
\newcommand{\Product}{\textsc{Linker}}

% ---------- 標題、頁首頁尾 ----------
\usepackage{titlesec}
\titleformat{\section}{\Large\bfseries\color{Brand}}{\thesection}{0.8em}{}
\titleformat{\subsection}{\large\bfseries}{\thesubsection}{0.6em}{}
\usepackage{fancyhdr}
\pagestyle{fancy}
\fancyhf{}
\lhead{\Product{} 學習歷程檔案}
\rhead{\thepage}

% ---------- 交叉引用 / 連結 ----------
\usepackage[hidelinks,unicode]{hyperref}
\hypersetup{colorlinks=true, linkcolor=Brand, urlcolor=Accent, citecolor=Brand}
\usepackage[nameinlink,noabbrev]{cleveref}
\crefname{figure}{圖}{圖}
\crefname{table}{表}{表}
\crefname{listing}{程式碼}{程式碼}

% ---------- 清單 / 表格 / 數學 ----------
\usepackage{enumitem}
\setlist{nosep}
\usepackage{booktabs}
\usepackage{amsmath,amssymb,mathtools}
\usepackage{array}
\newcolumntype{P}[1]{>{\raggedright\arraybackslash}p{#1}} % 不自動連字、左對齊段落欄

% ---------- 程式碼區塊(無需 shell-escape) ----------
\usepackage{listings}
\lstset{
  language=,
  basicstyle=\ttfamily\small,
  numbers=left, numberstyle=\footnotesize\color{SoftGray},
  showstringspaces=false, breaklines=true,
  frame=single, framerule=0pt, rulecolor=\color{LightBg},
  backgroundcolor=\color{LightBg},
  columns=fullflexible, keepspaces=true, upquote=true,
  keywordstyle=\bfseries\color{Accent},
  commentstyle=\itshape\color{SoftGray},
  stringstyle=\color{Brand},
  % 避免 PDF 複製被插空格(常見的 - / : 等)
  literate={-}{{-}}1 {/}{{/}}1 {:}{{:}}1 {_}{{\_}}1 {~}{{\textasciitilde}}1
}
\lstdefinestyle{py}{language=Python}
\lstdefinestyle{bash}{language=bash}

% ---------- tcolorbox 重點框 ----------
\usepackage[most]{tcolorbox}
\tcbset{colback=LightBg, colframe=Brand, arc=3mm, boxrule=.6pt}

% ---------- TikZ 圖(無模糊陰影,避免列印糊) ----------
\usepackage{tikz}
\usetikzlibrary{arrows.meta,positioning,fit,calc}
\tikzset{
  box/.style={
    draw=SoftGray, rounded corners, fill=LightBg,
    align=center, inner sep=6pt
  },
  pilarrow/.style={-Latex, thick, color=SoftGray}
}

% ---------- 本地化名稱 ----------
\renewcommand{\abstractname}{摘要}
\renewcommand{\contentsname}{目錄}
\renewcommand{\figurename}{圖}
\renewcommand{\tablename}{表}
\renewcommand{\lstlistingname}{程式碼}
\date{\the\year 年\the\month 月\the\day 日}

% ---------- 實用巨集 ----------
% 有圖就插圖,沒圖就不顯示(避免佔位框)
\newcommand{\MaybeFigure}[3][]{%
  \IfFileExists{#2}{%
    \begin{figure}[ht]
      \centering
      \includegraphics[width=\linewidth]{#2}
      \caption{#3}
    \end{figure}
  }{}
}

% 內文行內純文字程式碼(避免破行)
\newcommand{\code}[1]{\texttt{#1}}

% ---------- 文件資訊 ----------
\title{\textbf{從零到一:用 AI 協作打造智能英文學習平台}\\
\large 以 \Product{} 為例(FastAPI + Gemini 雙模型 + 個人化複習)}
\author{Max Chen \quad|\quad GitHub:\href{https://github.com/MaxChen228/Linker}{MaxChen228/Linker}}

\begin{document}
\maketitle

\begin{abstract}
\Product{} 是一個為個人英文學習打造的智能系統:即時批改、錯誤分類、知識點追蹤與複習排程。
本文件整合專案動機、設計決策與技術實作,並以圖解呈現「雙模型架構」與「知識管理」流程,
同時附上部署與操作速查。全文可在 Overleaf 以 XeLaTeX 直接編譯。
\end{abstract}

\tableofcontents
\bigskip

% =========================================
\section{為什麼做 \Product{}?(痛點與目標)}
\begin{tcolorbox}
\textbf{痛點}:寫句子缺乏道地度判斷、錯誤無即時與可追蹤回饋。\\
\textbf{目標}:\emph{即時批改} + \emph{精準分析} + \emph{個人化追蹤},不只標記對錯,更要說明「為什麼」和「如何更好」。
\end{tcolorbox}

\paragraph{設計原則}
\begin{itemize}
  \item \textbf{快速驗證}:先 CLI 後 Web,從最小可行功能起步(題目生成+批改)。
  \item \textbf{成本/品質兼顧}:採 \textbf{雙模型}——\emph{Flash} 出題、\emph{Pro} 批改,兼顧速度與準確。
  \item \textbf{可維護的 UI}:以設計令牌(Design Tokens)統一色彩、間距與字級,支援響應式。
\end{itemize}

% =========================================
\section{架構與流程(From CLI to Web)}
\subsection{模組演進}
\begin{itemize}
  \item \textbf{CLI 验證}:\code{linker\_cli.py}
  \item \textbf{Web 應用}:\code{web/main.py}(FastAPI + Jinja2):練習、知識點、文法頁。
  \item \textbf{核心模組}:\code{core/ai\_service.py}(雙模型)、\code{core/knowledge.py}(知識點管理)、\code{core/error\_types.py}(錯誤分類)、\code{core/logger.py}(集中日誌)。
\end{itemize}

\begin{figure}[ht]
  \centering
  \begin{tikzpicture}[node distance=10mm and 13mm]
    \node[box, fill=white, draw=Brand] (user) {使用者};
    \node[box, right=of user, xshift=6mm, text width=28mm] (ui) {Web UI\\(FastAPI + Jinja2)};
    \node[box, right=of ui, xshift=6mm, text width=28mm] (svc) {AI Service\\(\code{core/ai\_service.py})};
    \node[box, below=of svc, xshift=-22mm, text width=30mm] (gen) {出題模型\\Gemini 2.5 Flash\\\small temperature=1.0};
    \node[box, below=of svc, xshift=22mm, text width=30mm] (grade) {批改模型\\Gemini 2.5 Pro\\\small temperature=0.2};
    \node[box, right=of svc, xshift=8mm, text width=30mm] (extract) {錯誤分析/\\知識點提取};
    \node[box, right=of extract, xshift=8mm, text width=34mm] (km) {Knowledge Manager\\(\code{core/knowledge.py})};
    \node[box, below=of km, align=left, text width=36mm] (store) {\textbf{data/}\\knowledge.json\\practice\_log.json};

    \draw[pilarrow] (user) -- (ui);
    \draw[pilarrow] (ui) -- (svc);
    \draw[pilarrow] (svc) -- (gen);
    \draw[pilarrow] (svc) -- (grade);
    \draw[pilarrow] (gen) |- (extract);
    \draw[pilarrow] (grade) |- (extract);
    \draw[pilarrow] (extract) -- (km);
    \draw[pilarrow] (km) -- (store);
    \draw[pilarrow] (km.west) -- ++(-1.2,0) |- (ui.south);
  \end{tikzpicture}
  \caption{\Product{}:雙模型與知識管理資料流示意}
\end{figure}

% =========================================
\section{AI 雙模型策略(成本 × 品質)}
\begin{tcolorbox}
\textbf{策略}:\emph{出題} 使用 \textbf{Gemini 2.5 Flash}(快、創意、低成本),\emph{批改} 使用 \textbf{Gemini 2.5 Pro}(準確、專業、高品質),並調整溫度與取樣參數以平衡多樣性與穩定性。
\end{tcolorbox}

\subsection{核心程式片段(AI Service)}
\begin{lstlisting}[style=py, caption={雙模型初始化與參數(摘錄)}, label={lst:aisvc}]
# 出題模型(快速)- 提高溫度以增加變化
self.generate_model = genai.GenerativeModel(
    self.generate_model_name,
    generation_config=genai.GenerationConfig(
        response_mime_type="application/json",
        temperature=1.0,
        top_p=0.95,
        top_k=40,
    ),
)

# 批改模型(高品質)
self.grade_model = genai.GenerativeModel(
    self.grade_model_name,
    generation_config=genai.GenerationConfig(
        response_mime_type="application/json",
        temperature=0.2,
        top_p=0.9,
    ),
)
\end{lstlisting}

% =========================================
\section{知識點管理(錯誤類型 × 間隔重複)}
\subsection{四級錯誤分類}
\begin{table}[ht]
  \centering
  \begin{tabular}{P{3cm}P{10cm}}
    \toprule
    \textbf{類別} & \textbf{說明與示例} \\
    \midrule
    系統性(Systematic) & 可規則化避免的錯誤:時態、主被動、片語動詞用法等。 \\
    單一性(Isolated) & 需個別記憶:固定搭配、片語、慣用語。 \\
    可以更好(Enhancement) & 文法正確但不地道:自然度、語域、用字更精準。 \\
    其他(Other) & 漏譯、誤解、語意錯置等。 \\
    \bottomrule
  \end{tabular}
  \caption{錯誤分類設計}
\end{table}

\subsection{複習排程(艾賓浩斯導向)}
\begin{figure}[ht]
  \centering
  \begin{tikzpicture}[node distance=6mm]
    \foreach \i/\t in {0/Day 0,1/Day 1,2/Day 3,3/Day 7,4/Day 14,5/Day 30}{
      \node[box, minimum width=2.6cm] (d\i) at (2.9*\i,0) {\t};
      \ifnum\i>0 \draw[pilarrow] (d\the\numexpr\i-1\relax) -- (d\i); \fi
    }
  \end{tikzpicture}
  \caption{知識點複習時間點(示意,實際依掌握度動態調整)}
\end{figure}

% =========================================
\section{前端設計系統(Design Tokens)}
以設計令牌統一色彩、間距與字級,讓樣式跨頁一致、易擴充且利於重構。響應式布局支援桌機與行動。
\begin{tcolorbox}
\textbf{好處}:一致性、可維護、可替換品牌色;\quad
\textbf{做法}:集中於 \code{web/static/css/design-system/},頁面僅引用語意化類別。
\end{tcolorbox}

% =========================================
\section{成果與介面}
\subsection{核心頁面}
\begin{itemize}
  \item 練習:新題/複習、即時批改與解說。
  \item 知識點:卡片化瀏覽、依錯誤類型與掌握度篩選。
  \item 文法句型:內建查詢與示例。
\end{itemize}

% 若放了 figs/ui_practice.png 就會出圖;沒放就略過不顯示
\MaybeFigure[width=\linewidth]{ui_practice.png}{介面示例(練習頁)}

% =========================================
\section{部署與啟動(速查)}
\subsection{本機 Web}
\begin{lstlisting}[style=bash, caption={啟動服務(Linux/Mac)}]
# 1) 設定 API Key
export GEMINI_API_KEY="your-api-key"

# 2) 啟動服務
./run.sh

# 3) 瀏覽器開啟
# 注意:以下 URL 會正確顯示且可複製
# http://localhost:8000
\end{lstlisting}

\subsection{Docker}
\begin{lstlisting}[style=bash, caption={Docker 啟動}]
docker-compose up -d
# 之後訪問:
# http://localhost:8000
\end{lstlisting}

\subsection{雲端部署(Render)}
\begin{lstlisting}[style=bash, caption={Render 部署流程(摘要)}]
# 1) Fork 專案並連接 GitHub 倉庫
# 2) 在 Render 設定環境變數 GEMINI_API_KEY
# 3) 觸發部署完成後,即可對外提供服務
\end{lstlisting}

% =========================================
\section{反思與發展路線圖}
\subsection{經驗}
\begin{itemize}
  \item 與 AI 協作要 \emph{明確需求、質疑並共創}:快速原型 → 針對性優化。
  \item 設計系統讓前端在功能擴張時保持穩定與一致。
  \item 後端以事件日誌與知識點資料結構貫穿學習閉環,可量化成效。
\end{itemize}

\subsection{Roadmap}
\begin{itemize}
  \item \textbf{短期}:帳號系統與跨裝置同步。
  \item \textbf{中期}:JSON \textrightarrow{} SQLite 遷移,增強查詢與效能。
  \item \textbf{長期}:社群共學、分享知識點與筆記。
\end{itemize}

% =========================================
\section*{附錄 A:環境變數(示例)}
\begin{lstlisting}[style=bash]
# AI 模型設定
GEMINI_API_KEY=your-api-key
GEMINI_GENERATE_MODEL=gemini-2.5-flash
GEMINI_GRADE_MODEL=gemini-2.5-pro
# 日誌
LOG_LEVEL=INFO
LOG_TO_FILE=true
\end{lstlisting}

% ---- 備註:若想改用 minted(更漂亮的高亮) ----
% 1) Overleaf 專案 Compiler: XeLaTeX;在 Menu 勾選 -shell-escape
% 2) 把上方 listings 換成:
% \usepackage{minted}
% \setminted{fontsize=\small, breaklines, autogobble, frame=single, bgcolor=LightBg}
% 3) 將 \begin{lstlisting} 換成 \begin{minted}{python/bash} 即可

\end{document}
