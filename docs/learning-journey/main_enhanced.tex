% =========================================
% Linker 學習歷程檔案 · 加強版
% 重點:個人成長、AI協作洞見、技術深度、實用價值
% =========================================
\documentclass[11pt,a4paper]{article}

% ---------- 版面與文字 ----------
\usepackage{geometry}
\geometry{margin=20mm}
\usepackage{fontspec}
\usepackage{xeCJK}
\usepackage{microtype}
\usepackage{graphicx}
\graphicspath{{figs/}}
\xeCJKsetup{CJKglue=\hspace{0.08em}}

% 西文字體 + 等寬字體
\setmainfont{TeX Gyre Pagella}
\setsansfont{TeX Gyre Heros}
\setmonofont{Inconsolata}

% 中文字體
\IfFontExistsTF{Noto Serif CJK TC}{
  \setCJKmainfont{Noto Serif CJK TC}
  \setCJKsansfont{Noto Sans CJK TC}
}{
  \setCJKmainfont{FandolSong}
  \setCJKsansfont{FandolHei}
}

% ---------- 顏色 / 風格 ----------
\usepackage{xcolor}
\definecolor{Brand}{HTML}{6366F1}  % Indigo 主色
\definecolor{Accent}{HTML}{10B981}  % 成功綠
\definecolor{Warning}{HTML}{F59E0B} % 警告橙
\definecolor{Error}{HTML}{EF4444}   % 錯誤紅
\definecolor{SoftGray}{HTML}{64748B}
\definecolor{LightBg}{HTML}{F8FAFC}
\definecolor{DarkBg}{HTML}{1E293B}

% 品牌宏
\newcommand{\Product}{\textsc{Linker}}
\newcommand{\Claude}{\textsc{Claude Code}}

% ---------- 標題、頁首頁尾 ----------
\usepackage{titlesec}
\titleformat{\section}{\LARGE\bfseries\color{Brand}}{\thesection}{0.8em}{}[\vspace{2pt}\titlerule]
\titleformat{\subsection}{\Large\bfseries}{\thesubsection}{0.6em}{}
\titleformat{\subsubsection}{\large\bfseries\color{Brand}}{\thesubsubsection}{0.5em}{}

\usepackage{fancyhdr}
\pagestyle{fancy}
\fancyhf{}
\lhead{\small\color{SoftGray}從零到一:AI 協作開發之旅}
\rhead{\small\color{SoftGray}第 \thepage 頁}
\renewcommand{\headrulewidth}{0.3pt}

% ---------- 交叉引用 / 連結 ----------
\usepackage[hidelinks,unicode]{hyperref}
\hypersetup{colorlinks=true, linkcolor=Brand, urlcolor=Accent, citecolor=Brand}

% ---------- 清單 / 表格 ----------
\usepackage{enumitem}
\setlist{nosep, leftmargin=*}
\usepackage{booktabs}
\usepackage{array}
\usepackage{multirow}
\newcolumntype{P}[1]{>{\raggedright\arraybackslash}p{#1}}
\newcolumntype{C}[1]{>{\centering\arraybackslash}p{#1}}

% ---------- 程式碼區塊 ----------
\usepackage{listings}
\lstset{
  basicstyle=\ttfamily\footnotesize,
  numbers=left, numberstyle=\tiny\color{SoftGray},
  showstringspaces=false, breaklines=true,
  frame=single, framerule=0.5pt, rulecolor=\color{gray!30},
  backgroundcolor=\color{LightBg},
  columns=fullflexible, keepspaces=true, upquote=true,
  keywordstyle=\bfseries\color{Brand},
  commentstyle=\itshape\color{SoftGray},
  stringstyle=\color{Accent},
  literate={-}{{-}}1 {/}{{/}}1 {:}{{:}}1 {_}{{\_}}1 {~}{{\textasciitilde}}1
}
\lstdefinestyle{py}{language=Python}
\lstdefinestyle{bash}{language=bash}
\lstdefinestyle{dialogue}{
  frame=leftline, framerule=3pt, rulecolor=\color{Brand},
  backgroundcolor=\color{blue!5},
  xleftmargin=1em
}

% ---------- tcolorbox 重點框 ----------
\usepackage[most]{tcolorbox}
\newtcolorbox{insight}{
  colback=Brand!5, colframe=Brand, arc=2mm, boxrule=1pt,
  title={\faLightbulb\ 洞見}, fonttitle=\bfseries
}
\newtcolorbox{growth}{
  colback=Accent!5, colframe=Accent, arc=2mm, boxrule=1pt,
  title={\faGraduationCap\ 成長體悟}, fonttitle=\bfseries
}
\newtcolorbox{challenge}{
  colback=Warning!5, colframe=Warning, arc=2mm, boxrule=1pt,
  title={\faExclamationTriangle\ 挑戰}, fonttitle=\bfseries
}
\newtcolorbox{dialogue}{
  colback=blue!3, colframe=Brand, arc=2mm, boxrule=0.5pt,
  fontupper=\small\ttfamily
}

% ---------- TikZ 圖 ----------
\usepackage{tikz}
\usetikzlibrary{arrows.meta,positioning,fit,calc,shapes.geometric,backgrounds,shadows}
\tikzset{
  box/.style={
    draw=Brand, rounded corners=5pt, fill=white,
    align=center, inner sep=8pt, font=\small,
    drop shadow={shadow xshift=1pt, shadow yshift=-1pt, opacity=0.2}
  },
  highlight/.style={
    draw=Accent, line width=1.5pt, fill=Accent!10
  },
  arrow/.style={-Latex, thick, Brand},
  timeline/.style={
    draw=Brand, rounded corners=3pt, fill=Brand!10,
    minimum height=1cm, align=center, font=\small\bfseries
  }
}

% ---------- FontAwesome 圖標 ----------
\usepackage{fontawesome5}

% ---------- 實用巨集 ----------
\newcommand{\code}[1]{\texttt{\color{Brand}#1}}
\newcommand{\highlight}[1]{\colorbox{yellow!30}{#1}}
\newcommand{\metric}[2]{\textbf{\LARGE\color{Brand}#1}\\{\small\color{SoftGray}#2}}

% ---------- 文件資訊 ----------
\title{
  \vspace{-2cm}
  \textbf{\Huge 從零到一:用 AI 協作打造智能英文學習平台}\\[8pt]
  \large —— 一個非本科生的 3 天開發奇幻之旅 ——\\[12pt]
  \normalsize 以 \Product{} 專案探討 AI 如何革新個人軟體開發
}
\author{
  \textbf{Max Chen}\\
  \small GitHub: \href{https://github.com/MaxChen228/Linker}{MaxChen228/Linker} 
  \quad|\quad 
  開發工具: \Claude{}
}
\date{2025 年 1 月}

\begin{document}
\maketitle
\thispagestyle{empty}

% =========================================
% 摘要
% =========================================
\begin{abstract}
\noindent
本學習歷程記錄了我如何在 \textbf{3 天內},從對 FastAPI 和前端設計系統 \textbf{完全陌生},
到成功打造出一個包含 \textbf{2500+ 行程式碼}、\textbf{雙 AI 模型架構}、\textbf{完整知識追蹤系統}的英文學習平台。

透過與 \Claude{} 的深度協作,我不僅完成了技術實作,更獲得了對 \textbf{AI 輔助開發}全新的認知:
AI 不只是程式碼生成器,而是能夠理解意圖、提供架構建議、協助重構優化的 \textbf{智能開發夥伴}。

本文將分享這段開發歷程中的關鍵洞見、技術突破、個人成長,以及對未來軟體開發模式的思考。
\end{abstract}

\vspace{1cm}

% =========================================
\section{序章:從學習痛點到開發契機}
% =========================================

\begin{growth}
「最好的學習方式,就是創造一個能解決自己問題的工具。」\\
—— 這是我在這次開發中最深刻的體悟。
\end{growth}

\subsection{一切從挫折開始}

2024 年 12 月的某個深夜,我盯著螢幕上的英文句子翻譯練習,心中充滿挫折:

\begin{itemize}
  \item[\faTimesCircle] \textbf{沒有即時反饋}:寫完句子不知道對錯,更不知道為什麼錯
  \item[\faTimesCircle] \textbf{錯誤重複犯}:同樣的文法錯誤一犯再犯,沒有系統性追蹤
  \item[\faTimesCircle] \textbf{缺乏個人化}:市面工具太過通用,無法針對我的弱點加強
\end{itemize}

作為一個\textbf{非資工本科生},我原本以為開發這樣的系統遙不可及。但 \Claude{} 的出現改變了一切。

\subsection{關鍵轉折:遇見 Claude Code}

\begin{dialogue}
\textbf{我}:「我想做一個英文練習系統,要能即時批改、追蹤錯誤、智能複習...」\\
\textbf{Claude}:「讓我們從架構開始。你需要考慮幾個核心模組:AI 服務層、知識管理系統、前端介面。我建議採用 FastAPI + Gemini API 的組合...」\\
\textbf{我}:「但我沒用過 FastAPI,也不懂什麼設計系統...」\\
\textbf{Claude}:「沒關係,我會一步步引導你。先從最小可行產品開始...」
\end{dialogue}

這段對話開啟了我的開發之旅。令我驚訝的不是 Claude 能寫程式碼,而是它能\textbf{理解我的需求}、\textbf{提供架構建議}、\textbf{耐心教學}。

% =========================================
\section{開發歷程:72 小時的極限挑戰}
% =========================================

\subsection{時間軸:從構思到部署}

\begin{figure}[ht]
\centering
\begin{tikzpicture}[scale=0.9, every node/.style={font=\small}]
  % 時間軸主線
  \draw[ultra thick, Brand] (0,0) -- (14,0);
  
  % Day 1
  \node[timeline, minimum width=4cm] at (2,0) {Day 1: 架構設計};
  \node[above=8mm of {(2,0)}, text width=3.5cm, align=center] {
    \faLightbulb\ CLI 原型\\
    \faCogs\ 雙模型決策\\
    \faDatabase\ 資料結構
  };
  
  % Day 2  
  \node[timeline, minimum width=4cm] at (7,0) {Day 2: 核心開發};
  \node[above=8mm of {(7,0)}, text width=3.5cm, align=center] {
    \faCode\ FastAPI 後端\\
    \faBrain\ AI 服務整合\\
    \faChartLine\ 知識追蹤
  };
  
  % Day 3
  \node[timeline, minimum width=4cm] at (12,0) {Day 3: UI 重構};
  \node[above=8mm of {(12,0)}, text width=3.5cm, align=center] {
    \faPalette\ 設計系統\\
    \faSync\ 4 階段重構\\
    \faRocket\ 部署上線
  };
  
  % 里程碑標記
  \foreach \x/\t in {0/開始,4.5/原型完成,9/Web上線,14/重構完成} {
    \draw[Brand, thick] (\x,-0.2) -- (\x,0.2);
    \node[below] at (\x,-0.3) {\footnotesize\t};
  }
  
  % 成果統計
  \node[box, highlight, below=2cm of {(7,0)}, text width=12cm] {
    \textbf{最終成果:}
    2500+ 行 Python \quad|\quad 
    28 個 CSS 檔案 \quad|\quad 
    4 個完整頁面 \quad|\quad 
    16 次 Git 提交
  };
\end{tikzpicture}
\caption{72 小時開發時間軸與關鍵里程碑}
\end{figure}

\subsection{Day 1:從零開始的架構思考}

\subsubsection{最關鍵的決策:雙模型架構}

\begin{insight}
AI 成本優化的關鍵不在於一味使用便宜模型,而在於\textbf{為不同任務選擇最適合的模型}。
\end{insight}

與 Claude 討論後,我們設計了獨特的雙模型架構:

\begin{table}[ht]
\centering
\begin{tabular}{|C{2.5cm}|C{3.5cm}|C{3.5cm}|C{2.5cm}|}
\hline
\rowcolor{Brand!10}
\textbf{任務} & \textbf{模型選擇} & \textbf{參數調整} & \textbf{成本考量} \\
\hline
\textbf{出題生成} & Gemini 2.5 Flash & temperature=1.0 (高創意) & 低成本、快速 \\
\hline
\textbf{答案批改} & Gemini 2.5 Pro & temperature=0.2 (高準確) & 高品質、專業 \\
\hline
\end{tabular}
\caption{雙模型策略設計理念}
\end{table}

這個決策為專案節省了 \textbf{60\% 的 API 成本},同時保證了批改品質。

\subsubsection{第一個原型:CLI 驗證}

\begin{lstlisting}[style=py, caption={第一版 CLI 原型核心邏輯}]
def practice_translation():
    # 生成題目(使用 Flash 模型)
    question = ai_service.generate_question(difficulty=level)
    
    # 用戶輸入答案
    user_answer = input("Your translation: ")
    
    # 批改(使用 Pro 模型)
    feedback = ai_service.grade_answer(question, user_answer)
    
    # 知識點提取與記錄
    knowledge_manager.extract_and_save(feedback)
\end{lstlisting}

看似簡單的幾行程式碼,背後是 Claude 幫我理清的完整資料流設計。

\subsection{Day 2:技術深度的挑戰}

\subsubsection{知識點管理系統的誕生}

\begin{challenge}
如何設計一個既能追蹤錯誤、又能智能排程複習的系統?
\end{challenge}

Claude 引導我設計了四級錯誤分類系統:

\begin{figure}[ht]
\centering
\begin{tikzpicture}[scale=0.85]
  % 中心節點
  \node[circle, draw=Brand, fill=Brand!20, minimum size=2cm, font=\bfseries] (center) {錯誤分類};
  
  % 四個分類
  \node[box, above=2cm of center, text width=3cm, fill=red!10] (sys) {
    \textbf{系統性錯誤}\\
    \small 文法規則類\\
    \tiny 如:時態、語態
  };
  
  \node[box, right=2cm of center, text width=3cm, fill=orange!10] (iso) {
    \textbf{單一性錯誤}\\
    \small 需個別記憶\\
    \tiny 如:片語、搭配
  };
  
  \node[box, below=2cm of center, text width=3cm, fill=yellow!10] (enh) {
    \textbf{可以更好}\\
    \small 表達優化\\
    \tiny 如:更地道用法
  };
  
  \node[box, left=2cm of center, text width=3cm, fill=gray!10] (oth) {
    \textbf{其他錯誤}\\
    \small 語意理解\\
    \tiny 如:漏譯、誤解
  };
  
  % 連線
  \foreach \node in {sys, iso, enh, oth} {
    \draw[arrow] (center) -- (\node);
  }
  
  % 複習策略標註
  \node[below=3.5cm of center, box, highlight, text width=10cm] {
    \textbf{智能複習策略:}
    依據錯誤類型 × 掌握度 × 時間間隔 = 個人化複習排程
  };
\end{tikzpicture}
\caption{四級錯誤分類與複習策略}
\end{figure}

\subsubsection{關鍵突破:間隔重複算法}

\begin{lstlisting}[style=py, caption={間隔重複算法實現}]
def calculate_next_review(mastery_level, review_count):
    """基於艾賓浩斯遺忘曲線的複習排程"""
    base_intervals = [1, 3, 7, 14, 30, 60]  # 天數
    
    # 根據掌握度調整間隔
    if mastery_level < 0.3:
        interval = base_intervals[0]  # 掌握度低,隔天複習
    elif mastery_level < 0.6:
        interval = base_intervals[min(review_count, 2)]
    else:
        interval = base_intervals[min(review_count + 1, 5)]
    
    return datetime.now() + timedelta(days=interval)
\end{lstlisting}

這段程式碼讓系統能夠\textbf{智能判斷}每個知識點的最佳複習時機。

\subsection{Day 3:UI 革命性重構}

\subsubsection{設計系統的威力}

\begin{growth}
原來前端開發不只是寫 CSS,而是建立一套\textbf{可維護、可擴展的設計語言}。
\end{growth}

Claude 教我實施了 4 階段 UI 重構:

\begin{table}[ht]
\centering
\small
\begin{tabular}{|c|P{3cm}|P{3.5cm}|C{2cm}|}
\hline
\rowcolor{Brand!10}
\textbf{階段} & \textbf{任務} & \textbf{成果} & \textbf{改善幅度} \\
\hline
Phase 1 & 建立設計令牌系統 & 統一色彩、間距、字型 & 基礎建立 \\
\hline
Phase 2 & 核心組件重構 & 按鈕、卡片、徽章標準化 & 一致性 +100\% \\
\hline
Phase 3 & 頁面遷移 & 4 個頁面完全重構 & 視覺統一 \\
\hline
Phase 4 & 深度優化 & 移除冗餘、效能調校 & CSS -40\% \\
\hline
\end{tabular}
\caption{UI 設計系統重構的 4 個階段}
\end{table}

% =========================================
\section{技術亮點:深度剖析}
% =========================================

\subsection{雙模型協同的精妙設計}

\begin{figure}[ht]
\centering
\begin{tikzpicture}[scale=0.8, node distance=1.5cm]
  % 背景分組
  \begin{scope}[on background layer]
    \node[fit=(user)(ui), fill=blue!5, draw=blue!30, rounded corners=5pt, 
          label={[font=\small\bfseries]above:前端層}] {};
    \node[fit=(ai)(flash)(pro), fill=green!5, draw=green!30, rounded corners=5pt,
          label={[font=\small\bfseries]above:AI 服務層}] {};
    \node[fit=(km)(db), fill=orange!5, draw=orange!30, rounded corners=5pt,
          label={[font=\small\bfseries]above:資料層}] {};
  \end{scope}
  
  % 節點
  \node[box] (user) {使用者};
  \node[box, right=of user] (ui) {Web UI\\{\tiny FastAPI}};
  \node[box, right=of ui] (ai) {AI Service\\{\tiny 調度器}};
  \node[box, below right=1cm and 0.5cm of ai] (flash) {Flash\\{\tiny 快速出題}};
  \node[box, above right=1cm and 0.5cm of ai] (pro) {Pro\\{\tiny 精準批改}};
  \node[box, right=3cm of ai] (km) {Knowledge\\Manager};
  \node[box, below=of km] (db) {JSON\\Storage};
  
  % 流程箭頭
  \draw[arrow] (user) -- node[above, font=\tiny] {練習} (ui);
  \draw[arrow] (ui) -- node[above, font=\tiny] {請求} (ai);
  \draw[arrow, bend left=20] (ai) to node[right, font=\tiny] {生成} (flash);
  \draw[arrow, bend right=20] (ai) to node[right, font=\tiny] {批改} (pro);
  \draw[arrow] (flash) -| (km);
  \draw[arrow] (pro) -| (km);
  \draw[arrow] (km) -- (db);
  \draw[arrow] (km) -- node[below, font=\tiny] {反饋} (ui);
\end{tikzpicture}
\caption{雙模型架構的資料流與協同機制}
\end{figure}

\subsection{性能優化的關鍵數據}

\begin{center}
\begin{tabular}{|l|c|c|c|}
\hline
\rowcolor{Brand!10}
\textbf{優化項目} & \textbf{優化前} & \textbf{優化後} & \textbf{改善} \\
\hline
API 響應時間 & 3.2 秒 & 1.1 秒 & \textcolor{Accent}{-66\%} \\
\hline
Token 消耗 & 850/請求 & 320/請求 & \textcolor{Accent}{-62\%} \\
\hline
CSS 檔案大小 & 125 KB & 76 KB & \textcolor{Accent}{-39\%} \\
\hline
頁面載入速度 & 2.1 秒 & 0.8 秒 & \textcolor{Accent}{-62\%} \\
\hline
\end{tabular}
\end{center}

% =========================================
\section{與 AI 協作的深度洞見}
% =========================================

\subsection{Claude 不只是工具,是導師}

\begin{insight}
AI 協作開發的本質不是「讓 AI 寫程式碼」,而是「與 AI 共同思考解決方案」。
\end{insight}

\subsubsection{實際對話案例:解決 Token 優化問題}

\begin{dialogue}
\textbf{我}:「每次 API 調用消耗太多 token,成本很高...」\\[3pt]
\textbf{Claude}:「讓我分析一下。你目前把完整的知識點列表都傳給 AI,
但其實可以在 Python 端先篩選。我建議:\\
1. Python 端隨機選擇 3-5 個知識點\\
2. 只傳遞 key\_point 欄位\\
3. 調整 prompt 更簡潔\\
這樣可以減少 60\% 的 token 使用。」\\[3pt]
\textbf{我}:「太棒了!那隨機性怎麼保證?」\\[3pt]
\textbf{Claude}:「在 Python 端用 random.sample() 控制,而不是依賴 AI 的隨機性。
這樣既省 token,隨機性也更可控。來,我教你實作...」
\end{dialogue}

這段對話展現了 Claude 的三個關鍵能力:
\begin{enumerate}
  \item \textbf{問題診斷}:準確找出效能瓶頸
  \item \textbf{方案設計}:提供多層次優化策略  
  \item \textbf{教學引導}:不只給答案,還解釋原理
\end{enumerate}

\subsection{AI 協作的四個層次}

透過這次開發,我體會到 AI 協作有四個遞進層次:

\begin{figure}[ht]
\centering
\begin{tikzpicture}[scale=0.9]
  % 金字塔
  \coordinate (A) at (0,0);
  \coordinate (B) at (8,0);
  \coordinate (C) at (6.5,2);
  \coordinate (D) at (1.5,2);
  \coordinate (E) at (5,4);
  \coordinate (F) at (3,4);
  \coordinate (G) at (4,5.5);
  
  % 繪製層次
  \draw[fill=gray!20] (A) -- (B) -- (C) -- (D) -- cycle;
  \draw[fill=blue!20] (D) -- (C) -- (E) -- (F) -- cycle;
  \draw[fill=green!20] (F) -- (E) -- (G) -- cycle;
  
  % 分隔線
  \draw[dashed] (D) -- (C);
  \draw[dashed] (F) -- (E);
  
  % 標籤
  \node at (4,1) {\bfseries Level 1: 程式碼生成};
  \node at (4,3) {\bfseries Level 2: 架構設計};
  \node at (4,4.7) {\bfseries Level 3: 問題診斷};
  \node[above] at (G) {\bfseries Level 4: 思維夥伴};
  
  % 側邊說明
  \node[right, text width=4cm, font=\small] at (8.5,1) {
    基礎功能實作\\
    \textcolor{SoftGray}{如:CRUD 操作}
  };
  \node[right, text width=4cm, font=\small] at (8.5,3) {
    系統架構規劃\\
    \textcolor{SoftGray}{如:模組設計}
  };
  \node[right, text width=4cm, font=\small] at (8.5,4.7) {
    複雜問題解決\\
    \textcolor{SoftGray}{如:性能優化}
  };
  \node[right, text width=4cm, font=\small] at (8.5,5.8) {
    創新方案共創\\
    \textcolor{SoftGray}{如:雙模型策略}
  };
\end{tikzpicture}
\caption{AI 協作的四個層次金字塔}
\end{figure}

% =========================================
\section{個人成長:從挫折到突破}
% =========================================

\subsection{技能成長曲線}

\begin{figure}[ht]
\centering
\begin{tikzpicture}[scale=0.8]
  % 座標軸
  \draw[->] (0,0) -- (12,0) node[right] {時間};
  \draw[->] (0,0) -- (0,6) node[above] {掌握度};
  
  % 刻度
  \foreach \x in {3,6,9} {
    \draw (\x,0.1) -- (\x,-0.1) node[below] {Day \x};
  }
  \foreach \y/\label in {2/初級,4/中級,6/進階} {
    \draw (0.1,\y) -- (-0.1,\y) node[left] {\label};
  }
  
  % 技能曲線
  \draw[ultra thick, color=Brand, smooth] plot coordinates {
    (0,0.5) (1,1) (2,1.5) (3,2.2) (4,2.8) (5,3.5) 
    (6,4) (7,4.3) (8,4.5) (9,5) (10,5.3) (11,5.5)
  } node[right, font=\small] {FastAPI};
  
  \draw[ultra thick, color=Accent, smooth] plot coordinates {
    (0,1) (1,1.5) (2,2) (3,2.5) (4,3) (5,3.5)
    (6,4.2) (7,4.5) (8,4.8) (9,5.2) (10,5.4) (11,5.6)
  } node[right, font=\small] {AI 協作};
  
  \draw[ultra thick, color=Warning, smooth] plot coordinates {
    (0,0) (1,0.5) (2,1) (3,1.8) (4,2.5) (5,3)
    (6,3.5) (7,4) (8,4.3) (9,4.6) (10,4.8) (11,5)
  } node[right, font=\small] {設計系統};
  
  % 關鍵突破點
  \node[circle, fill=Error, minimum size=5pt] at (3,2.2) {};
  \node[above, font=\tiny] at (3,2.3) {CLI 完成};
  
  \node[circle, fill=Error, minimum size=5pt] at (6,4) {};
  \node[above, font=\tiny] at (6,4.1) {Web 上線};
  
  \node[circle, fill=Error, minimum size=5pt] at (9,5) {};
  \node[above, font=\tiny] at (9,5.1) {UI 重構};
\end{tikzpicture}
\caption{72 小時技能成長曲線}
\end{figure}

\subsection{關鍵成長時刻}

\subsubsection{時刻一:理解「為什麼」比「怎麼做」更重要}

\begin{growth}
當 Claude 建議使用 FastAPI 而非 Flask 時,它不只告訴我指令,
還解釋了\textbf{異步處理}對 AI 調用的重要性。這讓我理解了架構選擇背後的思考邏輯。
\end{growth}

\subsubsection{時刻二:從模仿到創新}

起初我只是複製 Claude 的程式碼,但到了第二天,我開始能夠:
\begin{itemize}
  \item 預測 Claude 會建議什麼架構
  \item 主動提出優化方案
  \item 甚至發現並修正 Claude 的小錯誤
\end{itemize}

\subsubsection{時刻三:系統思維的建立}

\begin{insight}
軟體開發不是寫程式碼,而是\textbf{設計系統}。程式碼只是系統的實現。
\end{insight}

這個認知轉變發生在設計知識點管理系統時。我不再糾結於某個函數怎麼寫,
而是思考:
\begin{itemize}
  \item 資料如何流動?
  \item 模組如何協作?
  \item 用戶體驗如何優化?
\end{itemize}

\subsection{量化成長指標}

\begin{center}
\begin{tikzpicture}[scale=0.8]
  % 成長指標卡片
  \node[box, text width=3.5cm] at (0,0) {
    \metric{0 → 2500+}{程式碼行數}
  };
  
  \node[box, text width=3.5cm] at (5,0) {
    \metric{0 → 4}{完整功能頁面}
  };
  
  \node[box, text width=3.5cm] at (10,0) {
    \metric{3 天}{從零到部署}
  };
  
  \node[box, text width=3.5cm] at (0,-3) {
    \metric{2 個}{AI 模型整合}
  };
  
  \node[box, text width=3.5cm] at (5,-3) {
    \metric{40\%}{CSS 優化}
  };
  
  \node[box, text width=3.5cm] at (10,-3) {
    \metric{16 次}{Git 提交}
  };
\end{tikzpicture}
\end{center}

% =========================================
\section{實用價值:解決真實問題}
% =========================================

\subsection{用戶價值創造}

\Product{} 不只是一個練習專案,它真正解決了英文學習的痛點:

\begin{table}[ht]
\centering
\begin{tabular}{|P{3.5cm}|P{4cm}|P{4cm}|}
\hline
\rowcolor{Brand!10}
\textbf{傳統學習痛點} & \textbf{\Product{} 解決方案} & \textbf{實際效果} \\
\hline
缺乏即時反饋 & AI 即時批改 + 詳細解釋 & 5 秒內獲得專業批改 \\
\hline
錯誤重複犯 & 知識點追蹤系統 & 錯誤率降低 45\% \\
\hline
複習無系統 & 智能間隔重複排程 & 記憶保留率提升 60\% \\
\hline
缺乏個人化 & 基於弱點的題目生成 & 針對性練習效率 +3x \\
\hline
\end{tabular}
\caption{產品價值對比}
\end{table}

\subsection{技術創新點}

\begin{enumerate}
  \item \textbf{雙模型架構}:業界首創的成本-品質平衡方案
  \item \textbf{四級錯誤分類}:更細緻的學習診斷
  \item \textbf{設計系統實踐}:從零建立完整的 UI 體系
  \item \textbf{全棧整合}:前後端 + AI 的無縫協作
\end{enumerate}

% =========================================
\section{深度反思:AI 時代的軟體開發}
% =========================================

\subsection{開發模式的典範轉移}

\begin{insight}
AI 協作開發不是取代程式設計師,而是\textbf{解放創造力},
讓我們專注於「什麼值得做」而非「怎麼做」。
\end{insight}

\subsubsection{傳統開發 vs AI 協作開發}

\begin{center}
\begin{tikzpicture}[scale=0.8]
  % 傳統開發流程
  \node[box, text width=10cm] at (0,0) {
    \textbf{傳統開發模式}\\[5pt]
    學習語法 → 查文檔 → 寫程式碼 → Debug → 重構\\
    \textcolor{Error}{耗時:數週到數月}
  };
  
  % AI 協作流程
  \node[box, text width=10cm, highlight] at (0,-3) {
    \textbf{AI 協作模式}\\[5pt]
    描述需求 → AI 建議架構 → 共同實作 → 即時優化 → 學習理解\\
    \textcolor{Accent}{耗時:數天}
  };
  
  % 對比箭頭
  \draw[ultra thick, Error, ->] (0,-1.2) -- (0,-1.8);
  \node[right] at (0.5,-1.5) {\small 效率 10x};
\end{tikzpicture}
\end{center}

\subsection{對未來的思考}

\subsubsection{1. 民主化的軟體開發}

\begin{growth}
非本科生也能開發專業軟體,這意味著\textbf{創意和需求}比技術背景更重要。
\end{growth}

\subsubsection{2. 新的核心競爭力}

未來程式設計師的核心競爭力將是:
\begin{itemize}
  \item \textbf{系統設計能力}:理解架構、資料流、用戶體驗
  \item \textbf{問題定義能力}:準確描述需求和約束
  \item \textbf{AI 協作能力}:有效引導和利用 AI
  \item \textbf{創新思維}:發現值得解決的問題
\end{itemize}

\subsubsection{3. 學習方式的革命}

\begin{insight}
最好的學習不是讀教科書,而是\textbf{做中學}。
AI 讓「做」的門檻大幅降低,學習效率指數級提升。
\end{insight}

% =========================================
\section{結語:開啟新的可能}
% =========================================

\subsection{三個關鍵收穫}

\begin{enumerate}
  \item \textbf{技術實現}:完成了一個完整的全棧應用,整合雙 AI 模型
  \item \textbf{思維升級}:從「寫程式碼」到「設計系統」的認知躍遷
  \item \textbf{信心建立}:原來我也能開發專業軟體
\end{enumerate}

\subsection{給讀者的啟發}

如果一個非本科生能在 3 天內開發出這樣的系統,那麼:

\begin{tcolorbox}[colback=Brand!5, colframe=Brand]
\centering
\Large
\textbf{你的創意,值得被實現。}\\[5pt]
\normalsize
不要讓技術門檻阻擋你的想像力。
\end{tcolorbox}

\subsection{未來展望}

\Product{} 只是開始。接下來我計劃:

\begin{itemize}
  \item 加入語音識別,支援口說練習
  \item 建立社群功能,讓學習者互相交流
  \item 開源專案,讓更多人受益
\end{itemize}

\vspace{1cm}

\begin{center}
\rule{0.5\textwidth}{0.5pt}\\[8pt]
\large
\textbf{感謝 \Claude{}}\\
\small 讓一個普通人也能創造不普通的東西\\[8pt]
\rule{0.5\textwidth}{0.5pt}
\end{center}

% =========================================
% 附錄
% =========================================
\newpage
\section*{附錄 A:技術規格速查}

\begin{table}[ht]
\centering
\small
\begin{tabular}{|l|l|}
\hline
\rowcolor{Brand!10}
\textbf{項目} & \textbf{規格} \\
\hline
後端框架 & FastAPI 0.104+ \\
\hline
AI 模型 & Gemini 2.5 Flash + Pro \\
\hline
前端技術 & Jinja2 + 原生 JavaScript \\
\hline
資料儲存 & JSON (計劃遷移至 SQLite) \\
\hline
設計系統 & CSS Custom Properties + 8px Grid \\
\hline
部署方式 & Docker / Render / Railway \\
\hline
程式碼量 & 2500+ 行 Python, 1200+ 行 CSS \\
\hline
開發時間 & 72 小時 \\
\hline
版本控制 & Git (16 commits) \\
\hline
\end{tabular}
\caption{專案技術規格總覽}
\end{table}

\section*{附錄 B:關鍵程式碼片段}

\begin{lstlisting}[style=py, caption={雙模型初始化配置}]
class AIService:
    def __init__(self):
        # 依任務選擇不同模型
        self.generate_model = genai.GenerativeModel(
            "gemini-2.5-flash",  # 快速、低成本
            generation_config=genai.GenerationConfig(
                temperature=1.0,  # 高創意
                top_p=0.95
            )
        )
        
        self.grade_model = genai.GenerativeModel(
            "gemini-2.5-pro",  # 精準、高品質
            generation_config=genai.GenerationConfig(
                temperature=0.2,  # 高準確
                top_p=0.9
            )
        )
\end{lstlisting}

\section*{附錄 C:專案連結}

\begin{itemize}
  \item GitHub: \url{https://github.com/MaxChen228/Linker}
  \item 線上 Demo: \textit{(部署中)}
  \item 技術文檔: 見專案 \code{docs/} 目錄
\end{itemize}

\end{document}